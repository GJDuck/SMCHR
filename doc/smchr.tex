\documentclass{article}
\usepackage{amsmath}
\usepackage{amssymb}
\usepackage{amsthm}
\usepackage{enumitem}

\newcommand{\bs}[0]{\symbol{92}~}

\title{The SMCHR System Version \input{VERSION}}
\author{Gregory J. Duck \\
\texttt{gregory@comp.nus.edu.sg}}
\date{}

\begin{document}

\maketitle
\newpage

\tableofcontents
\newpage

\section{Introduction}

The SMCHR system is an implementation of \emph{Satisfiability Modulo
Theories} (SMT), where the \emph{Theory} part can be implemented in
\emph{Constraint Handling Rules} (CHR).
In practice the SMCHR system is much more, and merges
several ideas into one unified platform, including:
\begin{itemize}[noitemsep,topsep=0pt,parsep=0pt,partopsep=0pt]
    \item \emph{Boolean Satisfiability} (SAT)
    \item \emph{Satisfiability Modulo Theories} (SMT)
    \item \emph{Constraint Programming} (CP)
    \item \emph{Lazy Clause Generation} (LCG) \emph{Finite Domain} (FD) Solving
    \item \emph{Constraint Handling Rules} (CHR)
\end{itemize}
Applications of the SMCHR system include:
\begin{itemize}[noitemsep,topsep=0pt,parsep=0pt,partopsep=0pt]
    \item Solving Constraint Satisfaction Problems (CSPs)
    \item Solving Boolean Satisfiability problems (SAT)
    \item Theory solver prototyping and implementation
    \item Theorem proving
    \item Program verification and analysis
\end{itemize}
And any other traditional SMT application.

The name ``\emph{SMCHR}'' stands for
``\emph{Satisfiability Modulo Constraint Handling Rules}'', and pays homage to
the original goal of the SMCHR system, namely
the integration of \emph{Constraint Handling Rules} (CHR)
into a \emph{Satisfiability Modulo Theories} (SMT) solver~\cite{smchr, smchr2}.
In addition to CHR, the SMCHR system support several \emph{built-in}
theory solvers, including, unification (equality) solvers,
finite domain (lazy clause generation) solvers,
linear arithmetic (Simplex-based) solvers, and
Separation-Logic style heap reasoning solvers for program
analysis/verification.
All built-in solvers are compatible with solvers implemented in Constraint
Handling Rules.

This document serves as the manual for the SMCHR system.
It describes how to download and install SMCHR, how to run SMCHR and
test queries for satisfiability, how to use the built-in solvers and
write new solvers using CHR.

%%%%%%%%%%%%%%%%%%%%%%%%%%%%%%%%%%%%%%%%%%%%%%%%%%%%%%%%%%%%%%%%%%%%%%%%%%%%
\section{The SMCHR Quick Start Guide}

\subsection{System Requirements}

The SMCHR system is supported on the following platforms:
\begin{center}
\begin{tabular}{|c|c|}
\hline
\emph{CPU} & \emph{Operating System} \\
\hline
\hline
& Linux \\
\texttt{x86\_64} & MacOSX \\
& Windows \\
\hline
\end{tabular}
\end{center}
32-bit CPUs are unlikely to ever be supported.

\subsection{Downloading and Installation}

The SMCHR system can be downloaded from the following URL:
\begin{quote}
    \texttt{http://www.comp.nus.edu.sg/\symbol{126}gregory/smchr/}
\end{quote}
SMCHR does not require any special installation -- simply unpack and run the
\texttt{smchr} executable.
For example, under Linux run:
\begin{verbatim}
    $ ./smchr.linux
\end{verbatim}
%To enable \texttt{libreadline} functionality, make sure that
%\texttt{libedit.so} appears in your dynamic link library path.
Press \emph{control-C} or \emph{control-D} to exit.

\subsection{Command Prompt}

The SMCHR system prompts the user for \emph{goal} formulae.
For example, if the user enters the goal:
\begin{verbatim}
    > x = y /\ x < y
\end{verbatim}
The SMCHR system will print the answer
\begin{verbatim}
    UNKNOWN
\end{verbatim}
indicating that (un)satisfiability is not known.

To determine that the goal is unsatisfiable, we need to load a
\emph{theory solver} that understands the standard interpretations of
the predicates the symbols (\texttt{=}), and (\texttt{<}).
Press \emph{control-C} to exit \texttt{smchr},
and re-run with the built-in \emph{linear arithmetic solver} loaded, e.g.
\begin{verbatim}
    $ ./smchr.linux --solver linear
    > x = y /\ x < y
\end{verbatim}
This time the SMCHR system will print that answer
\begin{verbatim}
    UNSAT
\end{verbatim}
indicating that the goal is unsatisfiable.

The SMCHR system only gives one of two answers:
\texttt{UNSAT} indicating that the goal is unsatisfiable, or
\texttt{UNKNOWN} indicating that unsatisfiability \emph{could not be proven}.
The latter can mean that either the goal is \emph{satisfiable}, or
that the solver is \emph{incomplete} and was unable to prove unsatisfiability.

\noindent \emph{NOTE}:
The SMCHR system assumes all solvers are incomplete, and therefore
will always answer \texttt{UNKNOWN} even if the goal is satisfiable.
However, if the user \emph{knows} that a given solver combination is
complete for a given goal $G$, then the answer \texttt{UNKNOWN} can
be re-interpreted as \texttt{SAT}.

%%%%%%%%%%%%%%%%%%%%%%%%%%%%%%%%%%%%%%%%%%%%%%%%%%%%%%%%%%%%%%%%%%%%%%%%%%%%%
\section{Goal Language}

\emph{Goals} are arbitrary (quantifier-free) formulae expressed in the
\emph{SMCHR goal language}.
The syntax is specified in Figure~\ref{fig:syntax}.
Here each operator has the standard associativity and precedence.
Note that the SMCHR system does not support SMT-LIB syntax.

\begin{figure}
\begin{center}
\begin{tabular}{|l|l|}
\hline
Syntax & Meaning \\
\hline
\hline
$\verb+x+$, $\verb+y+$, $..$ & Variable $x$, $y$, $..$ \\
$\verb+0+$, $\verb+1+$, $\verb+2+$, $..$ & Integer constants \\
$\verb+true+$, $\verb+false+$ & Boolean constants $\mathit{true}$,
    $\mathit{false}$ \\
$\verb+@f+$, $\verb+@g+$, $..$ & Atom constants $f$, $g$, $..$ \\
$\verb+"string"+$ & String constants \\
$\verb+nil+$ & Special \emph{nil} constant \\
\hline
$\verb+(+ E \verb+)+$ & Brackets $(E)$ \\
\hline
$\verb+not+~E$ & Negation $(\neg E)$ \\
$E~\verb+/\+~E$ & Conjunction $(E \wedge E)$ \\
$E~\verb+\/+~E$ & Disjunction $(E \vee E)$ \\
$E~\verb+->+~E$ & Implication $(E \rightarrow E)$ \\
$E~\verb+<->+~E$ & Iff $(E \leftrightarrow E)$ \\
$E~\verb+xor+~E$ & XOR $(E~\mathit{xor}~E)$ \\
\hline
$E~\verb+=+~E$ & Equality $(E = E)$ \\
$E~\verb+!=+~E$ & Disequality $(E \neq E)$ \\
$E~\verb+<+~E$ & Less-than $(E < E)$ \\
$E~\verb+<=+~E$ & Less-than-or-equal $(E \leq E)$ \\
$E~\verb+>+~E$ & Greater-than $(E > E)$ \\
$E~\verb+>=+~E$ & Greater-than-or-equal $(E \geq E)$ \\
\hline
$\verb+-+E$ & Negation $(-E)$ \\
$E~\verb-+-~E$ & Addition $(E + E)$ \\
$E~\verb+-+~E$ & Subtraction $(E - E)$ \\
$E~\verb+*+~E$ & Multiplication $(E \times E)$ \\
$E~\verb+/+~E$ & Division $(E \div E)$ \\
$E~\verb+^+~E$ & Exponentiation $(E^E)$ \\
\hline
$\verb+p(+E\verb+,+ ..\verb+,+E\verb+)+$ & Predicate/constraint
    $p(E, .., E)$ \\
\hline
$\verb+/*+~..~\verb+*/+$ & Multi-line comment \\
$\verb+//+~..$ & Single-line comment \\
\hline
\end{tabular}
\caption{SMCHR input language.\label{fig:syntax}}
\end{center}
\end{figure}

\noindent Some example goals include:
\vspace{-1em}
\begin{center}
\begin{tabular}{lll}
Goal: & & Meaning: \\
\verb+x /\ y+ & ~~~ & $x \wedge y$ (where $x$, $y$ are Boolean variables)
    \\
\verb?x + 4*y >= z \/ z < 7? & & $x + 4y \geq z \vee z < 7$ \\
\verb+p(x) /\ (q(x) \/ r(4, y))+ & & $p(x) \wedge (q(x) \vee r(4, y))$
\end{tabular}
\end{center}
Variable names may be upper or lower case.

\subsection{Built-in Constraints}

\begin{figure}
\begin{center}
\begin{tabular}{|l|l|l|}
\hline
Built-in & Meaning & Notes \\
\hline
\hline
$\mathtt{int\_eq\_c}(x, c)$           & $x = c$     & \\
$\mathtt{int\_eq}(x, y)$              & $x = y$     & $x \not\equiv y$\\
$\mathtt{int\_gt\_c}(x, c)$           & $x > c$     & \\
$\mathtt{int\_gt}(x, y)$              & $x > y$     & $x \not\equiv y$ \\
$\mathtt{int\_eq\_plus\_c}(x, y, c)$  & $x = y + c$ & $c \neq 0$ \\
$\mathtt{int\_eq\_plus}(x, y, z)$     & $x = y + z$ &
    $x \not\equiv y \not\equiv z$ \\
$\mathtt{int\_eq\_mul\_c}(x, y, c)$   & $x = c.y$   & $c \not\in \{0, 1\}$,
    $x \not\equiv y$ \\
$\mathtt{int\_eq\_mul}(x, y, z)$      & $x = y.z$   &
    $x \not\equiv y \not\equiv z$ \\
\hline
$\mathtt{nil\_eq\_c}(x, n)$           & $x = n$     & \\
$\mathtt{nil\_eq}(x, y)$              & $x = y$     & $x \not\equiv y$\\
\hline
$\mathtt{atom\_eq\_c}(x, a)$          & $x = a$     & \\
$\mathtt{atom\_eq}(x, y)$             & $x = y$     & $x \not\equiv y$\\
\hline
$\mathtt{str\_eq\_c}(x, s)$           & $x = s$     & \\
$\mathtt{str\_eq}(x, y)$              & $x = y$     & $x \not\equiv y$\\
\hline
\end{tabular}
\end{center}
\caption{Built-in constraints.\label{fig:builtin}}
\end{figure}

The SMCHR system supports several primitive built-in constraints as
shown in Figure~\ref{fig:builtin}.
Here $x \not\equiv y$ indicates that $x$ and $y$ must be distinct variables.

All SMCHR solvers operate over these built-in constraints.
The SMCHR front-end will automatically normalize the input goal formula
into a ``flat'' version in terms of these built-in constraints; in a way
such that satisfiability is preserved.

\subsection{Type-inst system}\label{sec:types}

\begin{figure}
\begin{center}
\begin{tabular}{|l|l|l|}
\hline
Type & Meaning & Notes \\
\hline
\hline
$\mathtt{nil}$    & Nil      & $\{\texttt{nil}\}$ \\
$\mathtt{bool}$   & Boolean  & $\{\texttt{true}, \texttt{false}\}$ \\
$\mathtt{atom}$   & Atom     & $\{\texttt{@f}, \texttt{@g}, ..\}$ \\
$\mathtt{num}$    & Number/Integer & $\mathbb{Z}$ \\
$\mathtt{string}$ & String   & $\{\texttt{""}, ..\}$ \\
$\mathtt{any}$    & Any type & \\
\hline
\end{tabular}
\end{center}
\caption{Built-in types and their meanings.\label{fig:types}}
\end{figure}

SMCHR supports a weak \emph{type-inst} system.
The standard type-names are shown in Figure~\ref{fig:types}.

Each function/predicate symbol has a type signature that must be respected
in order for a goal to be valid.
For example, the ($\texttt{+}$) function has the type
$\mathtt{num} \times \mathtt{num} \rightarrow \mathtt{num}$, so the goal
(\verb-true = "1" + nil-) will produce a type error.
In addition to these built-in type names, solvers may define new type names.

In addition to types, certain constraint symbols have an
\emph{instantiation} requirement, which specifies which arguments may or
may not be variables.
For example, the $\mathtt{int\_gt}(x, y)$ constraint requires that both 
arguments be variables of type $\mathtt{num}$.
Therefore, the goal \verb-int_gt(6, 3)- will be normalized to
(\verb-int_gt(x, y) /\ x = 6 /\ y = 3-) by introducing fresh variables
$x$ and $y$.
Likewise, the goal \verb-int_gt_c(x, y)- will produce an error because
it is impossible to normalize $y$ to a numeric constant.

%%%%%%%%%%%%%%%%%%%%%%%%%%%%%%%%%%%%%%%%%%%%%%%%%%%%%%%%%%%%%%%%%%%%%%%%%%%%
\section{Built-in Solvers} \label{sec:builtin}

The SMCHR system supports several \emph{built-in} solvers including:
\begin{itemize}[noitemsep,topsep=0pt,parsep=0pt,partopsep=0pt]
    \item $\mathtt{eq}$ -- An equality solver based on union-find.
    This solver is \emph{complete} for (dis)equality constraints.
    \item $\mathtt{linear}$ -- A linear arithmetic solver over the integers
    based on the Simplex algorithm over the rationals.
    This solver is \emph{incomplete} if there exists a rational solution
    for the given goal.
    \item $\mathtt{bounds}$ -- A simple bounds-propagation solver over the
    integers.
    This solver is \emph{incomplete}.
    \item $\mathtt{dom}$ -- A simple solver that interprets the constraint
    $\mathit{int\_dom}(x, l, u)$ as an integer domain/range constraint
    $x \in [l..u]$.
    This, in combination with the $\mathtt{bounds}$ solver, forms a
    \emph{Lazy Clause Generation} (LCG) finite domain solver.
    This solver is \emph{incomplete}.
    \item $\mathtt{heaps}$ -- A \emph{heap} solver for program
    reasoning based on some of the ideas from Separation Logic.
    See~\cite{heaps} for more information.
    This solver is \emph{complete}.
    \item $\mathtt{sat}$ -- The underlying SAT solver.
    This solver is \emph{complete} and is always enabled by default.
\end{itemize}
Any combination of these solvers can be loaded using the 
`\texttt{--solver $name$}' (or `\texttt{-s $name$}') command-line option.

%%%%%%%%%%%%%%%%%%%%%%%%%%%%%%%%%%%%%%%%%%%%%%%%%%%%%%%%%%%%%%%%%%%%%%%%%%%%
\section{CHR Solvers} \label{sec:chr}

The SMCHR system allows for the implementation of new theory solvers in
\emph{Constraint Handling Rules} (CHR).
These CHR solvers can then be loaded like any of the built-in solvers
using the `\texttt{--solver $name$}' command-line option.

\subsection{Syntax}

A CHR solver is file containing a set of \emph{rules} and
\emph{type-inst-declarations}.
CHR rules have one of three possible forms (\emph{simplification},
    \emph{propagation}, and \emph{simpagation} respectively):
\begin{align*}
\mathit{Head} & ~\texttt{<=>}~ [\mathit{Guard} ~\texttt{|}~]~\mathit{Body}
    \texttt{;} \\
\mathit{Head} & ~\texttt{==>}~ [\mathit{Guard} ~\texttt{|}~]~\mathit{Body}
    \texttt{;} \\
\mathit{Head}_1~\texttt{\bs}~\mathit{Head}_2 & ~\texttt{<=>}~
    [\mathit{Guard}~\texttt{|}~]~\mathit{Body} \texttt{;} 
\end{align*}
Both $\mathit{Head}$, $\mathit{Guard}$, and $\mathit{Body}$ are
\emph{conjunctions} (using \verb+/\+) of (negations of) constraints.
The $\mathit{Guard}$ is optional, as indicated by the square
parenthesis $[ .. ]$.

For example, the classic \emph{less-than-or-equal-to} CHR solver is
implemented as follows:
\begin{verbatim}
leq(x, x) <=> true;
leq(x, y) /\ leq(y, x) ==> x = y;
leq(x, y) /\ leq(y, z) ==> leq(x, z);
\end{verbatim}
Assuming this solver is saved in a file called \texttt{leq.chr}, then to
load the solver, pass the command-line option `\texttt{--solver leq.chr,eq}'
to the SMCHR system.
The `\texttt{eq}' part loads a built-in equality solver
(see Section~\ref{sec:builtin}).

\subsection{Negation}

The SMCHR allows the solver to directly manipulate \emph{negated constraints}.
For example, assuming that $\mathtt{leq}$ represents a total order, we can
write:
\begin{verbatim}
not leq(x, y) /\ not leq(y, x) ==> false;
not leq(x, y) /\ not leq(y, z) ==> not leq(x, z);
\end{verbatim}
Any combination of positive and negated constraints may appear in the same
rule.

\subsection{Disjunctive rules}

SMCHR also has limited support for disjunction in the rule body, i.e.
CHR$^\vee$.
For example:
\begin{verbatim}
p_or_q(x) <=> p(x) \/ q(x);
\end{verbatim}
SMCHR does not currently support mixing of conjunction and disjunction in
rule bodies.

\subsection{Guards}

SMCHR supports a primitive guard language as shown in Figure~\ref{fig:guards}.
\begin{figure}
\begin{center}
\begin{tabular}{|l|l|}
\hline
Guard & Meaning \\
\hline
\hline
$E~\texttt{\$=}~E$ & Numeric equality \\
$E~\texttt{\$!=}~E$ & Numeric disequality \\
$E~\texttt{\$<}~E$ & Numeric less-than \\
$E~\texttt{\$<=}~E$ & Numeric less-than-or-equals \\
$E~\texttt{\$>}~E$ & Numeric greater-than \\
$E~\texttt{\$>=}~E$ & Numeric greater-than-or-equals \\
$v~\texttt{:=}~E$ & Numeric assignment \\
\hline
\end{tabular}
\end{center}
\caption{Supported guards.\label{fig:guards}}
\end{figure}
Here $v$ is a \emph{variable} and $E$ is a \emph{numeric expression} of
the form:
\begin{align*}
E ::= \mathit{int}~|~v~|~E \texttt{+} E~|~E \texttt{-} E~|~
    E \texttt{*} E~|~E \texttt{/} E
\end{align*}
and has the usual meaning.

Note that guards only support primitive numeric tests and should not be
confused with arbitrary \emph{ask constraints}.

\subsection{Semantics}

The SMCHR system works by repeatedly applying CHR rules to the constraint
store until either (1) a fixed-point is reached (i.e. no more rules are
applicable), or (2) failure occurs.
The standard CHR semantics apply:
\emph{propagation rules} ($\mathit{Head}~\texttt{==>}~\mathit{Body}$)
\emph{add}
the constraints in $\mathit{Body}$ whenever constraints matching
$\mathit{Head}$ are found,
\emph{simplification rules} ($\mathit{Head}~\texttt{<=>}~\mathit{Body}$)
\emph{replace} the constraints matching $\mathit{Head}$ with 
the constraints in $\mathit{Body}$, and
\emph{simpagation rules}
($\mathit{Head}_1~\backslash~\mathit{Head}_2~\texttt{<=>}~\mathit{Body}$)
are a \emph{hybrid} between simplification and propagation rules, where
only the constraints matching $\mathit{Head}_2$ are replaced by
the constraints in $\mathit{Body}$.

\subsubsection*{Comparison with other CHR systems}

SMCHR follows the \emph{theoretical} operational semantics of CHR.
No assumptions should be made about the ordering of rule applications, etc.

Unlike other CHR systems, the SMCHR system uses \emph{set semantics} by
default.
It is therefore not necessary to specify a rule to enforce set
semantics\footnote{
In fact, such a rule is equivalent to (\texttt{leq(x, y) <=> true}) under
set semantics matching.}, e.g.:
\begin{verbatim}
leq(x, y) \ leq(x, y) <=> true;             // This rule is wrong!
\end{verbatim}
The SMCHR system also uses set semantics for \emph{matching}, meaning that
a constraint may match more than one occurrence in the head of a rule.
For example, the constraint $\mathtt{not}~\mathtt{leq}(a, a)$ will match
the rule:
\begin{verbatim}
not leq(x, y) /\ not leq(y, z) ==> false;
\end{verbatim}
with $x = a \wedge y = a \wedge z = a$.

The SMCHR system also treats deleted constraints differently.
A deleted constraint stays deleted ``forever'', i.e. it is not possible to
re-generated a copy of same constraint using a rule.
As a consequence, the following rule is always terminates in SMCHR:
\begin{verbatim}
p(x) <=> p(x);
\end{verbatim}

\subsection{CHR and equality}

The CHR engine relies on the `$\mathtt{eq}$' solver to determine if two
variables are equal.
However, including the `$\mathtt{eq}$' solver is optional.
For example, given
\begin{verbatim}
p(x) /\ q(x) ==> r(x)
\end{verbatim}
and the goal
\begin{verbatim}
> p(x) /\ q(y) /\ x = y
\end{verbatim}
then this rule will only match if the `$\mathtt{eq}$' solver was included.
Currently the CHR engine does not use other solvers, e.g. the
`$\mathtt{linear}$' solver, to determine if variables are equal.

\subsection{Type-Inst Declarations}

In addition to rules, the SMCHR system allows for the type-inst of each
constraint to be specified.
The general syntax for a \emph{type-inst} declaration is:
\begin{align*}
\texttt{type}~p([\texttt{var of}~]\mathit{type}_1, ..,
    [\texttt{var of}~]\mathit{type}_n)\texttt{;}
\end{align*}
where $\mathit{type}_i$ are type names such as those specified in
Section~\ref{sec:types}.
For example, the type-inst declaration of the built-in $\mathtt{int\_eq\_c}$
constraint is
\begin{verbatim}
type int_eq_c(var of num, num);
\end{verbatim}
The type declaration are used to check if the input goals are type-safe
with respect to the CHR-defined constraints.

The prefix `$\texttt{var of}$' specifies that the constraint expects the
argument to be a variable.
In absence of this prefix, the argument is be assumed to be a non-variable.
This affects how constraints are normalized.
If no type-inst declaration is present, the SMCHR system assumes a
default declaration where every argument has the type
`$\texttt{var of any}$'.

New type names can be implicitly declared by using a new type name in a
type-inst declaration.
For example, the type-inst declaration
\begin{verbatim}
type union(var of set, var of set, var of set);
\end{verbatim}
will implicitly define a new type named `$\mathtt{set}$'.

\subsection{Built-in constraints}

CHR rules can directly manipulate built-in constraints.
For example, the rule:
\begin{verbatim}
not int_eq_c(x, c) <=> d := c - 1 |
    int_gt_c(x, c) \/ not int_gt_c(x, d);
\end{verbatim}
will rewrite the built-in \emph{disequality} $x \neq c$ constraint
into a disjunction of \emph{inequalities} $x > c \vee x < c$.

A rule body may use equality $x = y$ syntax directly.
This is treated as syntactic sugar for a built-in
equality constraint of the appropriate type, e.g.
$\mathtt{int\_eq}(x, y)$, $\mathtt{int\_eq\_c}(x, y)$ for
integers.
Note that only equality has this special support.

%%%%%%%%%%%%%%%%%%%%%%%%%%%%%%%%%%%%%%%%%%%%%%%%%%%%%%%%%%%%%%%%%%%%%%%%%%%%%
\section{Debugging}

The SMCHR system includes a built-in interactive debugger.
The debugger is enabled via the `\texttt{-d}' command-line option, or
whenever the solver is interrupted (e.g. with Control-C).
The debugger works for all kinds of solvers, both CHR and built-in.

The debugger stops at various ``ports'' during the computation.
A debugger prompt of the following form is displayed:
\begin{tabbing}
\texttt{[S=$\mathit{STEP}$,L=$\mathit{LEVEL}$]~$\mathit{PORT}$:~$\mathit{CLAUSE}$>}
\end{tabbing}
where $\mathit{STEP}$ is the step number,
$\mathit{LEVEL}$ is the solver decision level,
$\mathit{PORT}$ is the port name, and
$\mathit{CLAUSE}$ is a clause representation of the port.
The supported ports include:
\begin{itemize}[noitemsep,topsep=0pt,parsep=0pt,partopsep=0pt]
\item \texttt{PROPAGATE}: When a new literal is set to true via propagation.
\item \texttt{FAIL}: When failure is detected.
\item \texttt{LEARN}: When a new clause is learnt during backtracking.
\item \texttt{SELECT}: When a new literal is selected during search.
\end{itemize}
Both \texttt{PROPAGATE} and \texttt{FAIL} come in two flavors:
\texttt{[T]} = propagation/failure by a theory solver;
\texttt{[S]} = propagation/failure by the SAT solver.

Several debugger commands are available.
These include:
\begin{itemize}[noitemsep,topsep=0pt,parsep=0pt,partopsep=0pt]
\item[-] \texttt{step} or \texttt{s}: move forward one step.
\item[-] \texttt{step $N$} or \texttt{s $N$}: move forward $N$ steps.
\item[-] \texttt{jump $N$} or \texttt{j $N$}: jump forward $N$ steps.
\item[-] \texttt{goto $N$} or \texttt{g $N$}: goto step $N$.
\item[-] \texttt{continue} or \texttt{c}: continue execution.
\item[-] \texttt{break} or \texttt{b}: clear the breakpoint.
\item[-] \texttt{break $S$} or \texttt{b $S$}: set a breakpoint on symbol $S$
    (name/arity).
\item[-] \texttt{abort} or \texttt{a}: abort execution.
\item[-] \texttt{dump} or \texttt{d}: dump the current solver state.
\item[-] \texttt{help} or \texttt{h} or \texttt{?}: print the help message.
\item[-] \texttt{quit} or \texttt{q}: quit the SMCHR system.
\end{itemize}

%%%%%%%%%%%%%%%%%%%%%%%%%%%%%%%%%%%%%%%%%%%%%%%%%%%%%%%%%%%%%%%%%%%%%%%%%%%%%
\section{Programming API}

The SMCHR system can be used as a library and integrated into other projects.
The programming API has deliberately being kept simple.
The main functions are:
\begin{itemize}[noitemsep,topsep=0pt,parsep=0pt,partopsep=0pt]
\item \verb+void smchr_init(void)+ -
    Initialize the SMCHR system.
\item \verb+bool smchr_load(const char *name)+ -
    Loads a solver with the name `\texttt{name}'.
\item \verb+term_t smchr_execute(term_t goal)+ -
    Executes a `\texttt{goal}' and returns the \emph{result}.
    The result is either:
    \begin{itemize}
        \item $\mathit{false}$ if `\texttt{goal}' is unsatisfiable;
        \item $\mathit{nil}$ if execution was aborted; or
        \item A conjunction of constraints for which the theory solvers
              cannot prove unsatisfiability.
    \end{itemize}
\end{itemize}
Terms can be (de)constructed using the functions defined in `\verb+term.h+'
(included by `\verb+smchr.h+') in the SMCHR code-base.
A brief summary of term construction is described below:
\begin{center}
\begin{tabular}{|l|l|}
\hline
\emph{Term} & \emph{Construction} \\
\hline
\hline
$\mathit{nil}$ & \verb+TERM_NIL+ \\
$\mathit{true}$ & \verb+TERM_TRUE+ \\
$\mathit{false}$ & \verb+TERM_FALSE+ \\
$0, 1, ..$ & \verb+term_int(0)+, \verb+term_int(1)+, $..$ \\
$x, y, ..$ & \verb+term_var(make_var("x"))+, \verb+term_var(make_var("y"))+,
    $..$ \\
$f(a_1, .., a_n)$ & \verb+term("f", a1, .., an)+ \\
\hline
\end{tabular}
\end{center}
For example, to use SMCHR to determine if the formula
$x + x < 2 \land y = x + 3 \land y > 4$ is satisfiable, then use:
\begin{verbatim}
    smchr_init();                       // Initialize SMCHR.
    smchr_load("linear");               // Load the linear solver.
    term_t x = term_var(make_var("x"));
    term_t y = term_var(make_var("y"));
    term_t c1 = term("<", term("+", x, x), term_int(2));
    term_t c2 = term("=", y, term("+", x, term_int(3)));
    term_t c3 = term(">", y, term_int(4));
    term_t goal = term("/\\", term("/\\", c1, c2), c3);
    term_t r = smchr_execute(goal);     // Execute the goal.
    if (r == TERM_FALSE)
        printf("UNSAT!\n");
    else
        printf("UNKNOWN!\n");
\end{verbatim}

%%%%%%%%%%%%%%%%%%%%%%%%%%%%%%%%%%%%%%%%%%%%%%%%%%%%%%%%%%%%%%%%%%%%%%%%%%%%%
\section{Example Solvers}

%%%%
\subsection{Set Solver}

We present a simple set solver based on
\emph{set element propagation}:
{
    \small
\begin{quote}
\begin{verbatim}
emp(s) /\ elem(s, x) ==> false;
one(s, x) ==> elem(s, x);
one(s, x) /\ elem(s, y) ==> x = y;
cup(s, s1, s2) /\ elem(s, x)  ==> elem(s1, x) \/ elem(s2, x);
cup(s, s1, s2) /\ elem(s1, x) ==> elem(s, x);
cup(s, s1, s2) /\ elem(s2, x) ==> elem(s, x);
cap(s, s1, s2) /\ elem(s, x)  ==> elem(s1, x) /\ elem(s2, x);
cap(s, s1, s2) /\ elem(s1, x) /\ elem(s2, x) ==> elem(s, x);
eq(s1, s2) /\ elem(s1, x) ==> elem(s2, x);
eq(s1, s2) /\ elem(s2, x) ==> elem(s1, x);
subseteq(s1, s2) /\ elem(s1, x) ==> elem(s2, x);
\end{verbatim}
\end{quote}
}
\noindent Here:
\begin{itemize}
\item $\mathtt{elem}(s, x)$ represents \emph{set element} $x \in s$;
\item $\mathtt{one}(s, x)$ is a \emph{singleton set} $s = \{x\}$;
\item $\mathtt{cup}(s, s_1, s_2)$ is \emph{set union} $s = s_1 \cup s_2$;
\item $\mathtt{cap}(s, s_1, s_2)$ is \emph{set intersection}
    $s = s_1 \cap s_2$;
\item $\mathtt{eq}(s_1, s_2)$ is \emph{set equality} $s_1 = s_2$; and
\item $\mathtt{subseteq}(s_1, s_2)$ is \emph{non-strict set subset}
    $s_1 \subseteq s_2$.
\end{itemize}
The constraint solver works by propagating set element through other
constraints.
Most of the rules are self-explanatory, e.g. the rule:
\begin{verbatim}
cup(s, s1, s2) /\ elem(s, x)  ==> elem(s1, x) \/ elem(s2, x);
\end{verbatim}
states that if $s = s_1 \cup s_2$ and $x \in s$, then
$x \in s_1$ or $x \in s_2$.

The solver will continue propagation until with unsatisfiability is detected,
or a fixed point is reached.
If the latter occurs, the $\mathtt{elem}$ constraints form a solution to the
original set constraints.

%%%
\subsection{Bounds Propagation Solver}

An implementation of a simple \emph{bounds propagation solver} is shown
in Figure~\ref{fig:bounds}.
This example shows all of the main features of SMCHR.

This solver under SMCHR is effectively a
\emph{lazy clause generation} finite domain (bounds propagation) solver
implemented in CHR.

\begin{figure}
{
\small
\begin{verbatim}
type lb(var of num, num);
type ub(var of num, num);

lb(x, l1) \ lb(x, l2) <=> l1 $> l2 | true;
ub(x, u1) \ ub(x, u2) <=> u1 $< u2 | true;
lb(x, l) /\ ub(x, u) <=> l $> u | false;
not lb(x, l) <=> u := l - 1 | ub(x, u);
not ub(x, u) <=> l := u + 1 | lb(x, l);

int_eq_c(x, c) ==> lb(x, c) /\ ub(x, c);
not int_eq_c(x, c) ==> u := c - 1 /\ l := c + 1 | ub(x, u) \/ lb(x, l);

int_gt_c(x, c) ==> l := c + 1 | lb(x, l);
not int_gt_c(x, u) ==> ub(x, u);

int_eq(x, y) /\ lb(x, lx) ==> lb(y, lx);
int_eq(x, y) /\ ub(x, ux) ==> ub(y, ux);
int_eq(x, y) /\ lb(y, ly) ==> lb(x, ly);
int_eq(x, y) /\ ub(y, uy) ==> ub(x, uy);
not int_eq(x, y) /\ lb(x, c) /\ ub(x, c) ==> u := c - 1 /\ l := c + 1 |
    ub(y, u) \/ lb(y, l);
not int_eq(x, y) /\ lb(y, c) /\ ub(y, c) ==> u := c - 1 /\ l := c + 1 |
    ub(x, u) \/ lb(x, l);

int_gt(x, y) /\ ub(x, ux) ==> uy := ux - 1 | ub(y, uy);
int_gt(x, y) /\ lb(y, ly) ==> lx := ly + 1 | lb(x, lx);

not int_gt(x, y) /\ lb(x, lx) ==> lb(y, lx);
not int_gt(x, y) /\ ub(y, uy) ==> ub(x, uy);

int_eq_plus_c(x, y, c) /\ lb(x, lx) ==> ly := lx - c | lb(y, ly);
int_eq_plus_c(x, y, c) /\ ub(x, ux) ==> uy := ux - c | ub(y, uy);
int_eq_plus_c(x, y, c) /\ lb(y, ly) ==> lx := ly + c | lb(x, lx);
int_eq_plus_c(x, y, c) /\ ub(y, uy) ==> ux := uy + c | ub(x, ux);
not int_eq_plus_c(x, y, d) /\ lb(x, c) /\ ub(x, c) ==>
        u := c - 1 - d /\ l := c + 1 - d |
    ub(y, u) \/ lb(y, l);
not int_eq_plus_c(x, y, d) /\ lb(y, c) /\ ub(y, c) ==>
        u := c - 1 + d /\ l := c + 1 + d |
    ub(x, u) \/ lb(x, l);
\end{verbatim}
}
\caption{Simple bounds propagation solver.\label{fig:bounds}}
\end{figure}

\bibliographystyle{plain}
\bibliography{smchr}

\end{document}

